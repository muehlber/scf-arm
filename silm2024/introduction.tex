
\section*{Preamble}

Confidential computing, with a focus on fortifying the integrity of code
and data actively in use, has prominently embraced the deployment of
\acp{TEE}, exemplified by technologies like ARM TrustZone. These
\acp{TEE} leverage hardware capabilities to establish secure enclaves for
applications, elevating the security posture. However, despite the robust
security framework provided by \acp{TEE}, the persistent threat of
side-channel attacks remains a formidable challenge, operating beyond
conventional threat model boundaries. Successful exploitation of these
attacks can compromise the default security assurances embedded in the
underlying hardware.

In the realm of embedded systems, ARM microcontrollers have emerged as
pivotal components. Their ubiquity in the embedded systems market can be
attributed to their versatility, energy efficiency, scalability, industry
support, and integration capabilities. These factors collectively make
ARM-based solutions a preferred choice for a wide range of applications,
contributing to their widespread adoption across various industries. Recent
strides in enhancing the security of ARM microcontrollers have seen the
integration of TrustZone, a technology designed to compartmentalize and
secure sensitive computations. Nevertheless, ARM microcontrollers equipped
with TrustZone have become enticing targets for side-channel attackers.
This chapter explores this duality — the amalgamation of heightened
security through TrustZone and the persistent vulnerability to side-channel
attacks.

Building upon the side-channel detection tool presented in chapter
\ref{ch:scfmsp}, this chapter takes a significant step forward to enhance
the precision of our previous static analysis tool. It leverages a
groundbreaking symbolic taint-tracking approach to conduct static analysis
on ARMv8-M binaries, proactively identifying both timing and storage
channels. We introduces \ac{SCF}\textsuperscript{ARM}, an advanced and
precise side-channel analysis tool designed to identify information leakage
stemming from timing side-channels, interrupt-latency attacks (commonly
known as Nemesis), novel DMA-based attacks (referred to as BUSted), and
unintended information flow within TrustZone applications tailored for ARM
microcontrollers.  The evaluation results affirm the robustness and
scalability of \ac{SCF}\textsuperscript{ARM} in detecting vulnerabilities
within realistic applications, thereby establishing its effectiveness as a
proactive measure for bolstering security in ARM-based confidential
computing environments.

\section{Introduction} \label{sect:intro3}

With the rapid proliferation of \ac{IoT} devices in various domains, such
as smart homes, healthcare, transportation, and industrial systems,
ensuring the security of these interconnected devices has become an utmost
concern. \ac{IoT} systems, consisting of embedded devices and networked
components, handle an abundance of sensitive data, making them prime
targets for malicious actors seeking to exploit vulnerabilities
\cite{IOTSecurity1, IOTSecurity2}. Among the multitude of security threats,
timing side-channel attacks have emerged as a significant and pervasive
challenge, leveraging timing variations to exploit vulnerabilities and
compromise the confidentiality and integrity of sensitive data
\cite{timingattack, Nemesis, Cache1, brumley2011remote, Travis, busted}.  

ARM family processors have emerged as the dominant choice for embedded
devices, capturing a substantial market share of over 60\%
\cite{arm_qualcomm}. To enhance security, ARM has incorporated TrustZone
\cite{TZM, DemystifyingAT}, a hardware-based security feature, into their
processors. TrustZone ensures the isolation of security-critical software
and data from the rest of the system, enabling secure execution of critical
tasks and protection of sensitive information. It achieves this by dividing
the processor into two separate and concurrent security realms or worlds:
the 'Normal World' and the 'Secure World.' These worlds operate
independently of each other, possessing distinct memory spaces and
execution environments. 

Here, developers often rely on the presumption that secrets are protected
within the secure world due to the processor's isolation guarantees.
However, extensive research \cite{surveyonTEE, DemystifyingAT, loadstep,
truspy, Bypassed, Qualcomm, busted} has revealed the potential
vulnerability of the TrustZone secure world to side channel attacks, which
can lead to the unintended disclosure of fine-grained secrets. 

%For instance, TruSpy \cite{truspy} exploits the cache contention between
%normal world and secure world to implement a timing-based cache
%side-channel attack and then extract a full 128-bit AES encryption key
%stored in the trusted environment. The research demonstrated that while
%the contents of the processor cache are safeguarded by the hardware
%isolation, the access pattern to these cache lines remains unprotected.
%Similarly, in \cite{Qualcomm}, researchers targeted Arm TrustZone in a
%malicious OS scenario. They leveraged the OS's capabilities to invoke
%interrupts and utilized the Prime+Probe \cite{primeandprobe} technique to
%recover a 256-bit private key from Qualcomm's ECDSA algorithm. %

The TrustZone technology employed in Armv8-M processors (such as
Cortex-M23/ M33/ M35P/ M55/ M85), do not claim to protect against side
channel attacks due to secret-dependent control flow with measurable timing
differences or secret-dependent memory access patterns \cite{armdeveloper}.
Additionally, it is important to note that TrustZone may not effectively
prevent secret leakage stemming from program implementation flaws, which
can arise from weaknesses in protocols or algorithms, as well as mistakes
made by developers. 

%As an example, let's consider an One-Time Password \gls{OTP} system
%implemented within the TrustZone environment \cite{trustotp}. In a secure
%and well-implemented \gls{OTP} system, once an \gls{OTP} is utilized, it
%should immediately become invalid and should not be stored in any
%accessible location. However, if the \gls{OTP}s are stored in an insecure
%manner, such as being logged or stored in plaintext on unprotected memory
%or external I/O, an unauthorized attacker who gains access to the system
%or the logs could retrieve the previously used \gls{OTP}s. %

Early detection of side channel attacks enables proactive mitigation
measures to be implemented. Over the years, researchers and practitioners
have proposed various approaches to analyze binary code, or source code
employing techniques such as symbolic execution \cite{binsec, pitchfork},
type systems \cite{scfmsp, MantelAVR, Agat, barthe2014system}, and machine
learning \cite{MLforSC}, among others \cite{timingattack}. These approaches
aim to identify and mitigate timing side channel vulnerabilities targeting
different architectures. However, each approach carries its own limitations
and strengths, necessitating a thorough exploration of the existing body of
work in this field (refer to Section \ref{sect:related3}).

In this chapter, we present an innovative automated approach utilizing symbolic execution-based analysis for the static verification of binaries targeting the ARM Cortex-M23 microcontroller. This approach capitalizes on the predictability of instruction execution times on Cortex-M23 microcontrollers. Our objective is to ensure the absence of timing side channel attacks, interrupt-latency attacks (such as Nemesis \cite{Nemesis}), DMA-based attacks (referred to as BUSted \cite{busted}), and detect any undesired explicit and implicit information flow, which is roughly equivalent to the concept of covert storage channels in later literature \cite{storagechannel}. This is particularly relevant in the context of applications that are compartmentalized into a security critical application part (such as managing and using cryptographic credentials) and a less critical part (such as sending and receiving network packets) to make use of the ARM TrustZone. 

Our proposed approach has been implemented in an automated tool,
\ac{SCF}\textsuperscript{ARM}, named after \cite{scfmsp}, to statically
verify ARMv8-M binaries against the aforementioned vulnerabilities. The
primary objective of \ac{SCF}\textsuperscript{ARM} is to track and monitor the flow of secret information between the TrustZone's secure world and the non-secure world, detecting and reporting any potential information leakages. To the best of our knowledge, this tool represents the first of its kind in performing static analysis on ARMv8-M binaries, addressing both timing and storage channels. 

To establish the efficacy of our approach, we conducted a thorough
evaluation of \ac{SCF}\textsuperscript{ARM}. This evaluation encompassed a
thoughtfully chosen set of benchmarks that spanned a diverse range of
programs, ranging from those with vulnerabilities to benign ones. These
benchmarks were intentionally designed to unveil both typical and
challenging structures of secret-dependent control flow, thereby offering a
thorough assessment of \ac{SCF}\textsuperscript{ARM}'s effectiveness. In summary, our contributions include:

\begin{itemize}
%
  \item{We have proposed a groundbreaking approach that harnesses the
capabilities of symbolic execution techniques to conduct sound information
flow analysis at the binary level. This approach is specifically tailored
for applications compartmentalized within ARM TrustZone, a widely adopted
security feature in commercial microcontrollers and mobile devices designed
to safeguard valuable and confidential data.}
%
  \item{To automate the process of checking ARMv8-M binaries and
identifying potential information leakages, we have implemented our novel
approach in a software tool called \ac{SCF}\textsuperscript{ARM}. Written
in Python, SCFARM efficiently carries out the analysis and provides
detailed reports on identified vulnerabilities. We have made both
\ac{SCF}\textsuperscript{ARM} and our benchmark datasets publicly available
on the GitHub repository at
\href{https://github.com/sepidehpouyan/ARM-SDFT}{https://github.com/sepidehpouyan/ARM-SDFT}}
%
  \item{We have successfully integrated static analysis techniques into
\ac{SCF}\textsuperscript{ARM}, enabling the detection of timing side
channel attacks, Nemesis attacks \cite{Nemesis}, BUSted attacks
\cite{busted}, and undesired direct and indirect information flow to
accessible and unprotected locations.}
%
  \item{To assess the effectiveness and scalability of
\ac{SCF}\textsuperscript{ARM}, we conducted a rigorous evaluation by
applying it to a set of vulnerable and benign programs targeting ARM
Cortex-M23. Our evaluation encompassed testing numerous scenarios to
analyze the accuracy of the tool in identifying information leakages and
its ability to handle larger codebases. The results demonstrated the high
precision and scalability of \ac{SCF}\textsuperscript{ARM}, validating its
utility in real-world security assessments.}
%
\end{itemize}

